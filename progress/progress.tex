% A ConTeXt file

%<a rel="license" href="http://creativecommons.org/licenses/by-sa/4.0/"><img 
%alt="Creative Commons License" style="border-width:0" 
%src="https://i.creativecommons.org/l/by-sa/4.0/88x31.png" /></a><br /><span 
%xmlns:dct="http://purl.org/dc/terms/" href="http://purl.org/dc/dcmitype/Text" 
%property="dct:title" rel="dct:type">A TeXer's Progress</span> by <a 
%xmlns:cc="http://creativecommons.org/ns#" 
%href="https://github.com/stephengaito" property="cc:attributionName" 
%rel="cc:attributionURL">Stephen Gaito</a> is licensed under a <a 
%rel="license" href="http://creativecommons.org/licenses/by-sa/4.0/">Creative 
%Commons Attribution-ShareAlike 4.0 International License</a>.<br />Based on a 
%work at <a xmlns:dct="http://purl.org/dc/terms/" 
%href="https://github.com/stephengaito/aTeXersProgress" 
%rel="dct:source">https://github.com/stephengaito/aTeXersProgress</a>. 

\starttext

\startsection[title={Introduction}]

A TeXer's Progress is a personal attempt, of an existing \LaTeX\ user, to 
understand how to write mathematical texts using \ConTeXt. We will produce 
three distinct documents. The first one, this document, will highlight the 
important points of this journey. The second document is a mathematical 
article about the chaotic dynamics of the H\'enon, Lozi and Two-shift maps, 
writen in \LaTeX. The third and last document, is the same mathematical 
article redone using \ConTeXt. 

\stopsection

\startsection[title={Where to put things}]

The first and most important problem to solve is where to put things. 
\ConTeXt\ has four main types of files, those which define Projects, 
Products, Components and Environments. This abundance of structure is very 
well suited to writing large complex multi-book projects. However, for our 
purposes we will only make use of Product, Component and Environment files. 

Each article will consist of a couple of major sections each of which will be 
written in their own file. The main article will then correspond to a Product 
which specifies which sections make up the article. Each major section will 
correspond to a Component. Environments are used to specify one or more 
article styles to be used by \ConTeXt\ when it lays-out the whole article. 

\ConTeXt's ability to manage complex documents is reflected in its ability to 
manage component files over multiple levels of directories. If a file, such 
as a Project, Product or Environment file is not found in the "current 
directory", \ConTeXt\ will search for this file in successive \emph{parent} 
directories until the search stops at the root of the filesystem. 

For our purposes we will use a simple two level structure. We will place all 
common files, such as the Environment files, in the top level directory. All 
of the Product and Component files for given article will be placed in the 
same subdirectory. 

\stopsection
\stoptext
