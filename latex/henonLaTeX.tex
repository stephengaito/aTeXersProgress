\documentclass{amsart}

% <a rel="license" href="http://creativecommons.org/licenses/by-sa/4.0/"><img alt="Creative Commons License" style="border-width:0" src="https://i.creativecommons.org/l/by-sa/4.0/88x31.png" /></a><br /><span xmlns:dct="http://purl.org/dc/terms/" href="http://purl.org/dc/dcmitype/Text" property="dct:title" rel="dct:type">A TeXer's Progress</span> by <a xmlns:cc="http://creativecommons.org/ns#" href="https://github.com/stephengaito" property="cc:attributionName" rel="cc:attributionURL">Stephen Gaito</a> is licensed under a <a rel="license" href="http://creativecommons.org/licenses/by-sa/4.0/">Creative Commons Attribution-ShareAlike 4.0 International License</a>.<br />Based on a work at <a xmlns:dct="http://purl.org/dc/terms/" href="https://github.com/stephengaito/aTeXersProgress" rel="dct:source">https://github.com/stephengaito/aTeXersProgress</a>.

\begin{document}

\section{Introduction}

The Henon map is a diffeomorphism of the plane developed by H\'enon to study chaotic dynamics. Since the Henon map is a diffeomorphism we should, in theory, be able to predict its behaviour for all iterates. In fact almost every orbit of the Henon map exhibits sensitive dependence on the initial conditions, otherwise known as Chaos.

The Lozi map is a piece-wise linear map of the plane developed by Lozi to study the behaviour of the Henon map.

The Two-shift is a map of the discrete topological space of either one or two sided sequences of two symbols, often taken to be $\{+1, -1\}, \{+, -\}$ or $\{0, 1\}$.

Our ultimate aim is to explore the relationship between these three maps in order to gain a better understanding of the structure of the chaotic dynamics of the Henon map.

\section{Henon map}

\end{document}